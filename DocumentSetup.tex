%Packages and document setup

%\usepackage{fullpage}
\usepackage[all]{xy}

\usepackage{graphicx} 	% Figures
\usepackage{subfigure}	% Subfigures
\usepackage{float}		% Better handling of Floats

\usepackage{amsmath}
\usepackage{amsthm}
\usepackage{amsfonts}
\usepackage{amssymb}
\usepackage{wasysym}

\usepackage[toc,page]{appendix}

\usepackage{relsize}
\usepackage{multicol}
\usepackage{multirow}

\usepackage{color}		% Colored text
\usepackage{tcolorbox}	% For paper review responses

	
\usepackage{latexsym}
\usepackage[psamsfonts]{eucal}

%\usepackage{cite}	
%\usepackage{citesort}


\usepackage[pdftex]{hyperref}
\hypersetup{pdfborder={0 0 0}}

\usepackage[margin=1in,paperwidth=8.5in,paperheight=11in]{geometry}



\usepackage{cancel} %So that I could use the cancel command.
\usepackage{pdfpages} % So that I could include pdf documents
\usepackage{makecell} 
\usepackage{soul} % For striking things out. \sout or \st
\usepackage{enumerate} 
%\setlength{\parindent}{0in}
\usepackage{amsthm} % For things such as Theorem
	\newtheorem{theorem}{Theorem}[chapter] %Increment theorems by chapter
	\newtheorem{proposition}{Proposition}[chapter]
	\newtheorem{lemma}{Lemma}[chapter]
	\newtheorem{corollary}{Corollary}[chapter]
	\newtheorem{definition}{Definition}[chapter]
	\newtheorem{remark}{Remark}[chapter]
	\newtheorem{question}{Question}[chapter]
	\newtheorem{problem}{Problem}
\usepackage{imakeidx} % For the index.
	\makeindex[intoc]
\usepackage{algorithm} %For nice looking algorithms
%\usepackage{algorithmicx}
\usepackage{algpseudocode} %For nice looking algorithms
\usepackage{array}
\usepackage{mathrsfs} % For script letter, \mathscr

%\usepackage{pgfplots}
\usepackage{tikz} % For commutative diagrams.	
\usepackage{tikz-cd} % For commutative diagrams.	
\usepackage{mathtools} % For mathclap (underbrace without space)
\usepackage{pgfgantt} %For Gantt charts.
\usepackage{xcolor,colortbl} 
	\definecolor{Grey}{gray}{0.9}
	\definecolor{White}{gray}{1}


\usepackage[T1]{fontenc}%For Biber and odd fonts
%\usepackage[T1]{fontenc}% So that listings gives the correct ASCII characters
\usepackage{listings} % For code
\lstset{
%basicstyle=\tiny,
basicstyle=\tiny,
keywordstyle=\color{blue},
commentstyle=\color{green},
stringstyle=\ttfamily,
numbers=left,%numberstyle=\noncopynumber,
columns=fullflexible
}



%\usepackage[toc]{glossaries}
%\makeglossaries
%\usepackage{epstopdf}
%\usepackage{pdflscape}



%%%%%%%%%%%%%%%%%%%%%%%%%%%%%%%%%%%%%%%%%%%%%%%%%%%%%%%%
%Document Look and Feel
\usepackage{fancyhdr}
\pagestyle{fancy}
\fancyhf{} % clear all header and footer fields
%%%\fancyfoot[R]{Choroszucha\ \thepage}
\lhead{\nouppercase\leftmark}
\rhead{{YOUR NAME HERE (DocumentSetup.tex)} \thepage}


%\fancyfoot{}
%%\renewcommand{\headrulewidth}{0pt}
%%\renewcommand{\footrulewidth}{0pt}


%\setlength{\paperwidth}{8.5in}
%\setlength{\paperheight}{11in}

%\setlength{\voffset}{-0.5in}
%\setlength{\hoffset}{0in}
%\setlength{\textwidth}{6.5in}%6in
%\setlength{\textheight}{9in}
%\setlength{\oddsidemargin}{0mm}
%\setlength{\evensidemargin}{0mm}
%\setlength{\topmargin}{0in}
%\setlength{\headheight}{0.25in}
%\setlength{\headsep}{0.25in}
%\setlength{\marginparsep}{5mm}
%\setlength{\marginparwidth}{0.75in}
%\setlength{\footskip}{0.25in}


\setlength{\headheight}{15pt}
\setlength\parskip{0.125in}
