\section{Moore - Principal Component Analysis In Linear Systems: Controllability, Observability, and Model Reduction - 1981}
\label{App:PaperNotes:MoorePCA1981}

This is paper \cite{Moore1981}.


\begin{proposition}[Proposition 4]
Define:
\[
S_F\isdef \left\{v:v\in {\rm im}(F(t)),\ t\in[t_1,t_2]\right\}
\]

Let $K$ be a fixed integer, $1\leq k \leq n$.  Over the class of piecewise continuous $F_A(t)$ satisfying $\dim\{S_{F_A}\}=K$, the residuals 
\begin{eqnarray}
JF&\isdef &\int_{t_1}^{t_2} ||F(t)-F_A(t)||_F^2 dt\\
JS&\isdef &\max_{||v||=1} \int_{t_1}^{t_2} ||v^T(F(t)-F_A(t))||^2 dt
\end{eqnarray}
are minimized with 
\begin{equation}
F_A(t)=F_K(t)\isdef \sum_{i=1}^k v_i f_i^T(t).
\end{equation}
The error residuals are 
\begin{eqnarray}
JF&=&\sum_{i=k+1}^n \sigma_i^2\\
JS&=&\sigma_{k+1}^2
\end{eqnarray}
\end{proposition}

\begin{proof}
It is easy to verify that $F_k(t)$ gives the stated residuals.  For an approximation $F_A$ which minimizes $JF$ or $JS$, it is necessarily true that $\int_{t_1}^{t_2}F_A(t)E_A^T(T)dt=0$ where $E_A(t)=F(t)-F_A(t)$.  Hence, for such an approximation 
\begin{equation}
W^2=\int_{t_1}^{t_2}F_A(t)F_A^T(t) dt+\int_{t_1}^{t_2}E_A(t)E_A^T(t)dt
\end{equation}
If $\int_{t_1}^{t_2}F_A(t)F_A^T(t) dt$ has rank $k$, it follows from perturbation properties of singular values that 
\begin{eqnarray}
JF&=& \tr\int_{t_1}^{t_2}E_A(t)E_A^T(t)dt\geq \sum_{i=k+1}^n \sigma_i^2\\
JS&=&\left\|\int_{t_1}^{t_2}E_A(t)E_A^T(t)dt\right\|\geq \sigma_{k+1}^2
\end{eqnarray}
\end{proof}





\begin{definition}[Output-Normal Form]
\label{Paper:Moore1981:OutputNorm}
A model $(A,B,C)$ is output-normal on $[0,T]$ if $W_c^2(P)=\Sigma^4$ and $W_o^2(P)=I$.
\end{definition}


\begin{definition}[Input-Normal Form]
\label{Paper:Moore1981:InputNorm}
A model $(A,B,C)$ is input-normal on $[0,T]$ if $W_c^2(P)=I$ and $W_o^2(P)=\Sigma^4$.
\end{definition}



\begin{definition}[Balanced Form]
\label{Paper:Moore1981:BalancedNorm}
A model $(A,B,C)$ is balanced on $[0,T]$ if $W_c^2(P)=\Sigma^2$ and $W_o^2(P)=\Sigma^2$.
\end{definition}